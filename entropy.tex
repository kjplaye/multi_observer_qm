\documentclass[12pt,a4paper]{article}
\usepackage{amsmath}
\usepackage{amsfonts}
\usepackage{amsthm}

\newtheorem{thm}{Theorem}
\newtheorem{lem}[thm]{Lemma}

\begin{document}

{\bf BIG TODO that we need for putting entropy in the ``weird.tex'' paper.  We need to know an interesting QM-generalization of ``zero-phase'' entropy.  It is necessarily not Von Neumann since pure states are always zero Von Neumann entropy.}

\section{Classical Information Theoretic Entropy}
We recall Shannon's classical entropy of the random variable $P$
\[
 H(P) := -\sum p_i \log(p_i)
\]
and relative entropy, or Kullback–Leibler divergence, of a model random variable $R$ with observations chosen from $P$
\[
D_{KL}(P||R) := -\sum p_i \log{\frac{r_i}{p_i}}.
\]

$H(P)$ is the average amount of ``surprise'' when using $P$, or the expected $P$ information
\[
H(P) = \mathbb{E}_P[-\log{P}].
\] 
For relative entropy we have another expected value appearing, the cross entropy, or expected $R$ information
\[
H(P||R) := \mathbb{E}_P[-\log{R}].
\]
We see how these are related in
\[
D_{KL}(P||R)  = H(P||R) -  H(P)  \ge 0
\]
where the last inequality is Gibbs' inequality.  It is a result of Gibbs that equality holds exactly when $R$ = $P$.

\subsection{Standard Results}

\begin{lem}
Let $f: X\rightarrow Y$ be a surjective map of finite random variables.  Then $H(X) \ge H(Y)$.  Furthermore, the following are equivalent
\begin{enumerate}
\item The fibers of $f$ each have at most one nonzero probability.
\item The fibers of $f$ each have zero entropy.
\item $H(X) = H(Y)$.
\end{enumerate}
\end{lem}
\begin{proof}
When $f$ is a bijection, $H(X) = H(Y)$, and we are done.  

We build up non-bijective $f$ as a series of maps
\[
  (A_0 = X) \rightarrow A_1 \rightarrow \cdots \rightarrow A_{k-1}  \rightarrow (A_k = Y)
\]
where $|A_i|$ = $|A_{i-1}|$ - 1.  In this way, we have a proof by induction on the size of $Y$.  That is, it suffices to prove the case where $|Y|$ = $|X|$ - 1 and apply it to the above $A_{\bullet}$ chain.

In this case there is a unique $y \in Y$ with non-singleton preimage.  That $y$ has exactly two unique preimages which we will call $x_1$ and $x_2$.
\[
  \{x_1,x_2\} \rightarrow \{y\}
\]

Let the probabilities of seeing $x_1$ and $x_2$ be $p_1$ and $p_2$ respectively.  The case where $p_1$ and $p_2$ are both zero results in $H(X) = H(Y)$ with the fiber condition satisfied, and we are done.

It remains to consider the case where at least one of the $p_i$ is nonzero.  The probability of seeing $y$ is $p_1 + p_2$ which we will also call $\alpha$ for short.  Let $q_i := p_i/\alpha$ and notice that $q_0 + q_1 = 1$.  We finally can compare the contributions to the difference of $H$ along the nontrivial fiber.
\[
\begin{split}
H(X) - H(Y) &= -p_1\log(p_1) - p_2 \log(p_2) + (p_1 + p_2) \log(p_1 + p_2) \\
            &= -\alpha q_1  \left[\log(q_1) + \log(\alpha)\right] - \alpha q_2 \left[\log(q_2) + \log(\alpha)\right] + \alpha \log(\alpha) \\
            &= \alpha (-q_1 \log(q_1) -q_2 \log(q_2)) \\
            &\ge 0
\end{split}
\]
which inductively shows that $H(X) \ge H(Y)$.  The inequality is equality exactly when $\{q_1,q_2\} = \{0,1\}$.  Inductively this finishes the claim about the fibers.
\end{proof}


\begin{lem}
A nontrivial projection measurement decreases the entropy.
\end{lem}
\begin{proof}
Consider a finite set $X$, the disjoint union $X$ = $X_1 \sqcup X_2$, and a measurement that projects onto $X_2$.  This sends $p_i \rightarrow 0$ for each $i \in X_1$ and renormalizes the elements of $X_2$ by $\sigma = \sum_{i \in X_2} p_i$.  Let $H(X)$ be the original entropy and let $H(\pi)$ be the entropy after the projection.
\[
\begin{split}
H(X) - H(\pi) &= \sum_{i \in X} - p_i \log(p_i) + \sum_{i \in X_2} \frac{p_i}{\sigma} \log\left(\frac{p_i}{\sigma}\right) \\
              &= \left(\sum_{i \in X_1} - p_i \log(p_i)\right) + \left(\sum_{i \in X_2} \left(-p_i + \frac{p_i}{\sigma}\right) \log(p_i) \right) - log(\sigma) \\
              &\ge 0
\end{split}
\]
Where the equality occurs exactly when $p_i = 0$ for all $i \in X_1$ and $\sigma = 1$, which is when the projection is trivial. 
\end{proof}


\begin{lem}
We choose observations according to a distribution $P$ and consider the Bayesian evidence towards another distribution $R$.  The inference, on average, prefers $P$ over $R$ with equality exactly when $R$ = $P$.
\end{lem}
\begin{proof}
Probabilities multiply so we need to work with logs to compute the expected values of the evidences.  Then the Lemma becomes exactly Gibbs' inequality.
\[
\mathbb{E}_P[-\log{R}] \ge \mathbb{E}_P[-\log{P}]
\]
\end{proof}

\end{document}
