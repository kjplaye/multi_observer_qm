\documentclass[12pt,a4paper]{article}
\usepackage[width=.75\textwidth]{caption}
\usepackage{graphicx}
\usepackage{authblk}
\usepackage{amsmath}
\usepackage{amsfonts}
\usepackage{braket}
\usepackage{epigraph}
\usepackage{amssymb}
%\usepackage{mathrsfs}
\usepackage[mathscr]{euscript}
\usepackage[top=2cm, bottom=2cm, left=2cm, right=2cm]{geometry}
\usepackage{fancyhdr}


\setlength{\epigraphwidth}{0.8\textwidth}

\pagestyle{fancy}
\begin{document}

%title and author details
\title{Multi-Observer Quantum Mechanics from Classical Bayesian Inference}
\author[1]{Kevin Player\footnote{kjplaye@gmail.com}}

\maketitle


\epigraph{I would not call [entanglement] {\bf one} but rather {\bf the} characteristic trait of quantum mechanics, the one that enforces its entire departure from classical lines of thought.}{Erwin Schrödinger}

\abstract{We present several quintessential quantum ideas and shed them in a classical light.  The ideas presented all have analogues in classical information theory.  For instance, entanglement is presented as a generalization of classical correlation.  We argue how quantum information theory should be understood as a generalization of classical information theory.  In fact, classical information theory is embedded in quantum information as zero phase wavefunctions.  We employ this embedding to highlight the importance of a new multi-observer version of quantum mechanics. Finally, we outline an experiment to test the existance of our multi-observer theory.}

\section{Quantum Weirdness List}
If you ask a physicist on the street what kinds of things are uniquely quantum mechanical they might just pick an item from this list

\begin{enumerate}
\item entanglement
\item wave function collapse
\item the measurement problem
\item tensor product of ensembles
\end{enumerate}

Entanglement is surely a uniquely quantum item with no classical analog.  The wave function collapsing is according to Penrose \cite{penrose}, an unsettlingly nonlinear mystery.  The measurement problem is aptly named.  And unlike the 8-dimensional 8-bit classical ensemble, 8 qbits require 256 dimensions from a tensor product.

\section{Some Thoughts}
With 8 coins in hand, we start some thought experiments.
\subsection{Configurations}
Let us try to describe the situation once we throw the 8 coins.  Let
\[
x_i \in \{T,F\}
\]
be the outcome for the throws $i$ = $1,\cdots,8$.  We have 8 ``dimensions'' worth of information to describe the 8 coins.  Lets move on and generalize.
  
\subsection{Classical Information Theory}
\subsubsection{Tensor Product for an Ensamble}
We want to describe a general knowledge statement about the coins.  This will actually involve more than 8 dimensions.  We let
\[
   p_0,\cdots,p_{255} \in \mathbb{R}^+
\]
with $\sum p_i = 1$.  Each $p_i$ is the probability of the coins being in a configuration state given by the binary expansion of $i$.  For instance $p_{71}$ is the probability of seeing $FTFFFTTT$. Notice that we have not moved to quantum information theory, but we are already motivated to use 256 dimensions instead of 8.  This can be thought of as a tensor product.  So cross ``tensor product of ensembles'' off the weirdness list.

\subsubsection{Bayesian Projections as Classical Wave Function Collapse}
The measurement of classical information is decidedly not weird, so this section should be familiar.  We don't always get to measure the exact configuration.  For instance, we might only get to know that the first coin is $T$ and that the 2nd and 3rd coins are the same.  This cooresponds to a subset of $\{0,\cdots,255\}$ which we will call $S$.
\[
   S = \{s \in \{0,\cdots,255\} | b_1(s) = T, b_2(s) = b_3(s) \}
\]
where $b_i(s)$ is the value $x_i$.  Someone would measure in this case that the state was in $S$.

\subsubsection{Bayesian Inference as General Classical Wave Function Collapse}
A more general type measurement is a probabilistic measurement.  Someone could learn that there is a 95\% chance that the state is in $S$.  In full generality we will call such a probablistic observation $\mathcal{O}$.  We can figure out how to update our knowledge statement, $p_i$, using a relative\footnote{Here the $P(\mathcal{O})$ cancels out and is wrapped up in the normalization. This omition reflects its non-physicality.} version of Bayes's rule
\[
  \frac{\hat{p_i}}{\hat{p_j}} = \underbrace{\frac{P(i | \mathcal{O})}{P(j | \mathcal{O})}}_\text{Posterior}
                              = \underbrace{\frac{P(\mathcal{O} | i)}{P(\mathcal{O}|j)}}_\text{Bayes Factor}  \hspace{0.07 in}  \underbrace{\frac{p_i}{p_j}}_\text{Prior}
\]
Pulling out the Bayes factor we find that we just multiply by the likelihood and re-normalize.
\[
  \hat{p_k} =  P(\mathcal{O} | k) \hspace{0.07 in} p_k
\]
A special case are projections $P(\mathcal{O} | k) \in \{0,1\}$, like the 100\% $S$ case above.  We call these Bayesian projections.

\subsubsection{Classical Correlation as Classical Entanglement}
Finally, we can also consider distributions such as the two coin distribution $p_0,p_1,p_2,p_3$
\[
p_i = 
\left\{
\begin{split}
\frac{1}{2} & \mbox{ if } i \in \{0,3\}\\
0 &\mbox{ else }
\end{split}
\right.
\]
We have $FF$ or $TT$ with equal probabilities.  If I give the first coin to Alice and the second coin to Bob then we have a classical correlation.  If Bob finds that the first coin is $T$ then we know that Alice will also find $T$.  This should all seem very reasonable; classical correlation is not weird.

\subsection{Quantum Information Theory}
\subsubsection{Classical to Quantum Embedding}
General quantum information\footnote{We are purposely leaving out density matrices and POVMs and just dealing with pure states.} about 8-qbits can be expressed as a wave function.
\[
   q_0,\cdots,q_{255} \in \mathbb{C}
\]
with $\sum |q_i|^2 = 1$.  The Born rule is ostensibly a map $q_i \rightarrow |q_i|^2 = p_i$ to classical probability.  Here we can instrument all of the classical information theoretic constructs by restricting the phase\footnote{We consider negative values as 180 degrees out of phase, so zero phase means non-negative real.} of $q_i$ to be zero.  We can map backward $p_i \rightarrow \sqrt{p_i} = q_i$; which commutes with the Born rule:

{
\renewcommand{\arraystretch}{0.1}
\[
\text{(Quantum)} \hspace{0.3 in}
\mathbb{C}^n \begin{array}{c} \twoheadrightarrow \\ \hookleftarrow \end{array}
(\mathbb{R}_{\ge 0})^n
\hspace{0.3 in} \text{(Classical)} 
\]
}

So quantum information theory needs to generalize the classical.  In particular, quantum measurement and entanglement restrict to classical measurement and classical correlation for zero phase.

\subsubsection{Bayesian Projection}
Without loss of generality, we motivate the generalization of Bayesian projection with an example.

Within the embedding we define a projection operator $\pi$ which projects onto the space generated by the subset $S$, from the previous subsection
\[
\pi = \sum_{i \in S} \ket{i} \bra{i}.
\]
Then let $A$ be any operator with an eigenspace, with eigenvalue $\lambda$, equal to the image of $\pi$.  An observation of $\lambda$ would correspond to an application of $\pi$, which is Bayesian projection in the classical picture.

\subsubsection{Bayesian Inference}
Finally, consider, in the fashion of \cite{nielsenchuang}, a more general measurement with matrix $M_\mathcal{O}$ which is diagonal with entries.
\[
   M_\mathcal{O}(i,i) = \sqrt{P(\mathcal{O} | i)}
\]
This implements classical Bayesian inference in the quantum realm.


\subsubsection{ENTROPY}
SPEAK TO ENTROPY\footnote{Von Neumann entropy is left out of this note since it is not a generalization of classical entropy in the manner presented in this note.  This is because the zero phase classical wavefunctions are pure states which all have zero Von Neumann entropy.}

\section{Overview of the Generalization}

The quantum mechanical concepts mentioned in Section 1 are a direct generalization of their classical information theoretic analogs.  This is not to say that they are not weird, but to say that they need to be generalizations of the classical concepts.  Wave function collapsing should generalize Bayesian inference and classical measurement.  Entanglement should be a generalization of classical correlation.  Some other classical ideas motivate new ways of doing quantum mechanics, which we will see these in the next sections.

\section{Multi-Observer Quantum Mechanics}
We wrap-up by focusing on quantum wave function collapse as a generalization of classical Bayesian inference.  We can implement the classical Bayesian inference within zero phase quantum mechanics as outlined above.  So the classical Bayesian theory has a direct tie in; that observation and measurement occur in tandem with a change in knowledge\footnote{This change in knowledge is tangible, it always occurs as a transfer of matter/energy from the environment to the observer \cite{thrust}.}.  

In the classical theory, knowledge is local to the observer\footnote{Note that current quantum theory is a theory of one observer, usually the experimenter in a lab.  An immediate subject is quantum key distribution(QKD), which requires at least three observers, Alice, Bob, and Eve; the security of QKD may depend on multi-observer quantum ontology.}, multiple observers each have their own knowledge.  For instance, Wigner's friend and Wigner each have their own classical knowledge.  It would seem then that Wigner's friend and Wigner must have different wavefunctions as well, if they are to be generalizing classical knowledge.

\section{Prediction and Experiment}

We predict that subjective multi-observer quantum mechanics will be required for a proper generalization of classical knowledge.  We propose an experiment where we inject an ``observer'' into a quantum eraser\footnote{Here we probe the boundaries of what constitutes an observer and a measurement.  We claim that all versions of observer and measurement will be detectable in this experiment whenever it can be done.}.  The injected observer will have to be a small apparatus that is
\begin{itemize}
   \item able to record a measurement $\psi$.
   \item able to forget the measurement $\psi$.
   \item able to demonstrate that a measurement was made with a record $\rho$.
\end{itemize}
Let $\phi$ be the lab technician's wavefunction.  In $\phi$ the apparatus is entangled with the subject particles in the eraser experiment.  We need to find out that there is another wavefunction in play that is not just part of $\phi$.  The key here is that the apparatus is able to demonstrate that it ``knew'' something, or in other words that the other wavefunction $\psi$ existed, using $\rho$.

The apparatus can not keep $\psi$ for proper erasure.  Observed erasure, via an interference pattern, proves that $\psi$ is not a part of $\phi$.  Finally, there should be a record $\rho$ in the apparatus that it did at one point in time record a measurement $\psi$.  Reciept of the report $\rho$ and eraser interference pattern together show that $\psi$ and $\phi$ were necessarily part of a two-observer system.

\section{Acknowledgments}
Thanks to Erik Ferragut for useful discussions.

\bibliographystyle{ieeetr}
\bibliography{bibliography}

\end{document}
