\documentclass[12pt,a4paper]{article}
\usepackage[width=.75\textwidth]{caption}
\usepackage{graphicx}
\usepackage{authblk}
\usepackage{amsmath}
\usepackage{amsfonts}
\usepackage{braket}
%\usepackage{mathrsfs}
\usepackage[mathscr]{euscript}
\usepackage[top=2cm, bottom=2cm, left=2cm, right=2cm]{geometry}
\usepackage{fancyhdr}

\pagestyle{fancy}
\begin{document}

%title and author details
\title{What is so Weird About Quantum Mechanics?}
\author[1]{Kevin Player\footnote{kjplaye@gmail.com}}

\maketitle

\abstract{We present several quintessential quantum ideas and shed them in a new, old, classical light.}

\section{The Weirdo List}
If you ask a physicist on the street what kinds of things are uniquely quantum mechanical they might just pick an item from this list

\begin{enumerate}
\item entanglement
\item wave function collapse
\item the measurement problem
\item tensor product of ensembles
\end{enumerate}

Entanglement is surely a uniquely quantum item with no classical analog.  The wave function collapsing is according to Penrose \cite{penrose}, an unsettlingly nonlinear mystery.  The measurement problem is aptly named.  And unlike the classical ensembles, 8 qbits require 256 dimensions from a tensor product, not just 8.

\section{Some Thoughts}
With 8 coins in hand, we start some thought experiments.
\subsection{Configurations}
Let us try to describe the situation once we throw the 8 coins.  Let
\[
x_i \in \{T,F\}
\]
be the outcome for the throws $i$ = $1,\cdots,8$.  We have 8 ``dimensions'' worth of information to describe the 8 coins.  Lets move on and generalize.
  
\subsection{Classical Information Theory}

We want to describe a general knowledge statement about the coins.  This will actually involve more than 8 dimensions.  We let
\[
   p_0,\cdots,p_{255} \in \mathbb{R}^+
\]
with $\sum p_i = 1$.  Each $p_i$ is the probability of the coins being in a binary state given by $i$.  For instance $p_{71}$ is the probability of seeing $FTFFFTTT$. Notice that we have not moved to quantum information theory, but we are already motivated to use 256 dimensions instead of 8.  This can be thought of as a tensor product.  So cross ``tensor product of ensembles'' off the weirdness list.

The measurement of classical information is decidedly not weird.  We don't in fact have to measure the exact configuration.  For instance, we might just get to know that the first coin is $T$ and that the 2nd and 3rd coins are the same.  This cooresponds to a subset of $\{0,\cdots,255\}$ which we will call $S$.
\[
   S = \{s \in \{0,\cdots,255\} | b_1(s) = T, b_2(s) = b_3(s) \}
\]
where $b_i(s)$ is the value $x_i$.  Someone could measure this by being told that the state was in $S$.

A more general type measurement is a probabilistic measurement.  Someone could be told that there is a 95\% chance that the state is in $S$.  In full generality we will call such a probablistic observation $\mathcal{O}$.  We can figure out how to update our knowledge statement, $p_i$, using a relative\footnote{Here the $P(\mathcal{O})$ cancels out and is wrapped up in the normalization. This omition reflects its non-physicality.} version of Bayes's rule
\[
  \frac{\hat{p_i}}{\hat{p_j}} = \underbrace{\frac{P(i | \mathcal{O})}{P(j | \mathcal{O})}}_\text{Posterior}
                              = \underbrace{\frac{P(\mathcal{O} | i)}{P(\mathcal{O}|j)}}_\text{Bayes Factor}  \hspace{0.07 in}  \underbrace{\frac{p_j}{p_j}}_\text{Prior}
\]
Pulling out the Bayes factor we find that we just multiply by the likelihood and re-normalize.
\[
  \hat{p_i} =  P(\mathcal{O} | i) \hspace{0.07 in} p_i
\]
A special case are projections $P(\mathcal{O} | i) \in \{0,1\}$, like the 100\% $S$ case above.  We call these Bayesian projections.

Finally, we can also consider distributions such as the two coin distribution
\[
p_i = 
\left\{
\begin{split}
\frac{1}{2} & \mbox{ if } i \in \{0,3\}\\
0 &\mbox{ else }
\end{split}
\right.
\]
Looking at just the first two coins, we have $FF$ or $TT$ with equal probabilities.  If I give the first coin to Alice and the second coin to Bob then we have a classical correlation.  If Bob finds that the first coin is $T$ then we know that Alice will also find $T$.  This should all seem very reasonable; classical correlation is not weird.

\subsection{Quantum Information Theory}
General quantum information\footnote{We are purposely leaving out density matrices and POVMs.} about 8-qbits can be expressed as a wave function.
\[
   q_0,\cdots,q_{255} \in \mathbb{C}
\]
with $\sum |q_i|^2 = 1$.  The Born rule is ostensibly a map $q_i \rightarrow |q_i|^2 = p_i$ to classical probability.  Here we can instrument all of the classical information theoretic constructs by restricting the phase of $q_i$.  For instance, we can map backward $p_i \rightarrow \sqrt{p_i} = q_i$.  So quantum information theory needs to generalize the classical.  In particular, quantum measurement and entanglement restrict to classical measurement and classical correlation for zero phase.

An example of this is to define a projection operator $\pi$ which projects onto the space generated by the subset $S$, from the previous subsection.
\[
\pi = \sum_{i \in S} \ket{i} \bra{i}
\]
Then to let $A$ be any operator with an eigenspace, with eigenvalue $\lambda$, equal to the image of $\pi$.  Then an observation of $\lambda$ would correspond to an application of $\pi$, which is Bayesian projection in the classical picture.

Finally, consider, in the fashion of \cite{nielsenchuang}, a more general measurement with matrix $M_\mathcal{O}$ which is diagonal with entries.
\[
   M_\mathcal{O}(i,i) = \sqrt{P(\mathcal{O} | i)}
\]
This implements classical Bayesian inference in the quantum realm.
   
\section{Conclusion}

It is the author's belief that the quantum mechanical concepts mentioned in section 1 are a direct generalization of their classical information theoretic analogs.  This is not to say that they are not weird, but to say that they need to be generalizations of the classical concepts.  Wave function collapsing should generalize Bayesian inference\footnote{QBism may be invoked here.} and classical measurement.  Entanglement should be a generalization of classical correlation.  These ideas have direct classical motivations, only after acknowledging this can we appreciate the weirdness that remains.

\section{Acknowledgments}
Thanks to Erik Ferragut for useful discussions.

\bibliographystyle{ieeetr}
\bibliography{bibliography}

\end{document}
